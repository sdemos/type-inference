\documentclass[12pt]{article}

\usepackage[letterpaper, margin=1.4in]{geometry}
\linespread{1.3}

\title{A Formalization of ``A Rewriting Semantics for Type Inference'' in Coq}
\author{Stephen Demos}

\begin{document}

\maketitle

The project that I worked on for the Software Foundations independent study was
to formalize the descriptions of the rewriting semantics described in the paper
``A Rewriting Semantics for Type Inference'' by George Kuan, David MacQueen,
and Robert Bruce Findler.

The idea presented in the paper is that instead of teaching type inference to
students as the classical recursive typing relation, it would be more effective
to teach a rewriting style type inference. The authors argue that this approach
is more natural and intuitive, and so it doesn't take as long for students to
understand, and then once they understand the intuition, professors can move on
to the more widely used typing standards.

In the paper, the authors describe the rules for rewriting relations for three
different languages, SLTC, ULC (where the type annotation on the bound
parameter is erased), and SLTC with let-polymorphism. In each section of the
paper, the authors list some interesting theorems about their rewriting
relation, as well as a sketch of the proof for each of them.

Most of my time on the project was spend on the formalization of the first
section of the paper, which went over the rewriting semantics for the STLC.
This included writing the relations, translating the theorems written in the
paper to theorems that Coq could understand, and attempting to prove them.

I went through several approaches for this portion. The main differences had to
do with how I represented the types and terms in the STLC. The crux of the
problem comes from the way that the paper describes the syntax. The authors use
two different expression representations. First they show the classic
description of the STLC, with separate types and terms. Then they present a new
expression type, which has types and terms combined, for the rewriting
semantics. References to their relations use these two representations
interchangeably, because they have the same syntax. However, Coq is not a fan
of this kind of implicit switching between types, and so machinery has to be
built up that is not reflected in the paper. 

The first approach that I took was the most obvious. For this approach, I just
wrote out all of the relations as they were described in the paper. This
approach is clean looking in the definitions section of the Coq script, but the
problem becomes obvious as soon as you try to write the Coq theorems that are
analogous to the presented theorems.

The second approach is a modification of the first one. This approach included
the functions \texttt{inj}, which converted terms to mixed expressions, and
\texttt{proj}, which converted mixed expressions to an Option that contained
STLC types, and nothing if the mixed expression contained more than just types.
This approach is in the file \texttt{stlc.v}.

The second approach allows for a Coq theorem that is closer to the original
theorem. The only difference is that it includes explicit conversions. There
are a couple of problems with this approach. First, this approach would be much
more tedious to expand to larger languages. The types and terms of the STLC are
repeated several times throughout the script. This isn't as much of a problem
for this specific application, because this is more of a proof-of-concept than
an extensible platform for language development, but it is something that is
important to keep in mind when designing these sorts of systems. Second, having
the conversions laying around makes the proof more difficult. 

The third approach that I attempted makes the Coq theorem look almost exactly
like the stated theorem in the paper. The trick in this approach was that I
only had one representation of types and terms of the STLC. The only types were
mixed expressions, and I had several propositions that checked if a mixed
expression was a type, a term, a value, etc. These predicates were used when
building the relations, as well as a quick little check at the beginning of the
theorems, which was the only difference.

This approach allowed for much clearer theorems. However, it can be difficult
to make sure that all of the constructors that need predicates actually have
them. Normally the work of making sure all of the constructors have the correct
type of data coming in is machine checked, however, this approach removes the
ability for Coq to see that in exchange for more flexibility.

For the proofs themselves, the issues I ran into were centered around the
substitution. I have not been able to figure out how to get the substitution
out of the picture so that I can use the induction hypothesis. Part of the
difficulty, I suspect, comes from that although the final outcome of the
classical typing relation and the rewriting relation are the same, they don't
take even close to the same kinds or number of steps to get there. This makes
it more difficult to actually state and prove their equivalence.

There was a difference between the Coq proof and the proof sketch in the paper
that I found interesting as well. In the paper, for the first theorem, they say
that the proof goes through easily with induction on the typing derivation.
When this is attempted in Coq, however, the proof gets stuck on the first case
of proving the typing derivation for a variable. This is because variables
don't have a way to step to anything in the rewriting semantics. This seems to
be a symptom of the problem I was describing earlier, where the rewriting
relation and typing derivation don't take the same path to get to their
solution, even though the solution is the same. This can be solved by inducting
on the expression instead of the typing derivation, and then using inversion on
the typing derivation to clear out the impossible cases that come out of that.

Probably the biggest lesson that I got from doing this project is that
formalizing something in a language like Coq can be much different than
formalizing it in the real world. Coq is a lot more rigid, because you are
trying to express your proofs in a set language instead of convincing another
person. The Coq approach gives much more confidence in the outcome than you
might have otherwise, but it also requires you to be much more explicit about
the behaviors that you are expecting from your relations and theorems than you
might otherwise have to be. This gave me insight into what elements of a formal
system to watch out for when I am thinking about formalizing something in Coq
or another proof assistant. 

\end{document}
